% !TeX spellcheck = en_US
\documentclass[a4paper]{scrartcl}

\usepackage[utf8]{inputenc}
\usepackage[english]{babel}
\usepackage[T1]{fontenc}
\usepackage{lmodern}
\usepackage{amsmath}
\usepackage{amssymb}
\usepackage{pdflscape}
\usepackage{geometry}
\usepackage{xcolor}
\usepackage{graphicx}
\usepackage{todonotes}
\setlength{\parindent}{0pt}

\usepackage{biblatex}
\addbibresource{references.bib}


%\geometry{a4paper, top=25mm, left=30mm, right=20mm, bottom=30mm,
%headsep=10mm, footskip=12mm}

\newcommand{\itab}[1]{\hspace{0em}\rlap{#1}}
\newcommand{\tab}[1]{\hspace{.2\textwidth}\rlap{#1}}
\newcommand{\asd}[1]{\textbf{#1}}

\title{Concepts for Digram types}
\author{Matthias Dürksen}
\date{\today}
 

\begin{document}
\maketitle
\section{Graph Definition}
 Ein gerichteter Hypergraph $G$ ist ein Tupel $(V,E)$. Dabei ist $V$ eine Menge von Knoten und $E$ eine Menge von Kanten.
 Dabei haben die Knoten und Kanten optional Label. Dabei beschreibt die Labelfunktion $l_N: V \to \Sigma$ zu 
 
\section{Related Work}

\subsection{Grammar-based Graph Compression}

\subsection{Maneth}



\end{document}